\usepackage[nonumberlist]{glossaries}
\renewcommand*{\glsclearpage}{}

\newglossaryentry{Dominion}{
	name=Dominion,
	description={Dominion ist ein Kartenspiel, das 2008 von Rio Grande Games
	veröffentlicht wurde}
}
\newglossaryentry{Lobby}{
	name=Lobby,
	description={Als Lobby wird der Raum bezeichnet, in dem sich die Benutzer
	vor dem Start eines Spiels versammeln und in dem das Spiel konfiguriert
	werden kann}
}
\newglossaryentry{Bot}{
	name=Bot,
	description={Ein Bot ist ein Teilnehmer an einem Spiel, der computergesteuert
	nach einem vordefinierten Verhaltensmuster agiert}
}
\newglossaryentry{Host}{
	name=Host,
	description={Der Host einer Lobby ist derjenige, der die Konfigurationen für
	das zu startende Spiel vornehmen kann. In der Regel handelt es sich dabei um
	den Ersteller der Lobby}
}
\newglossaryentry{EventBus}{
	name=Event-Bus,
	description={Der Event-Bus ist Teil der Guava-Bibliothek von Google und wird
	insbesondere für die Kommunikation zwischen dem Server und dem Client eingesetzt}
}
\newglossaryentry{Message}{
	name=Message,
	description={Eine Message ist eine Nachricht, die der Server an einen oder mehrere
	Clients verschickt, ohne dass diese vorher explizit danach gefragt haben}
}
\newglossaryentry{Request}{
	name=Request,
	description={Ein Request wird vom Client an den Server geschickt, um eine bestimmte
	Aktion auszulösen}
}
\newglossaryentry{Response}{
	name=Response,
	description={Ein Response ist eine unmittelbare Antwort an genau den Client, der vorher
	einen Request versendet hat}
}
\newglossaryentry{Event}{
	name=Event,
	description={Ein Event ist eine Nachricht, die lediglich client-intern verarbeitet wird}
}
\newglossaryentry{Benutzer}{
	name=Benutzer,
	description={Der Benutzer ist die Person, die den Client gestartet hat}
}
\newglossaryentry{Spieler}{
	name=Spieler,
	description={Ein Spieler ist im Allgemeinen ein Benutzer, der einer Lobby beigetreten ist
	oder sich aktiv in einer Spielrunde befindet}
}
\newglossaryentry{Zug}{
	name=Zug,
	description={Ein Zug eines Spielers umfasst die Aktions-, Kauf- und Aufräumphase. Mit einem Zug
	ist also die gesamte Zeit gemeint, in der ein Spieler aktiv ist, bis der aktive Spieler
	zum nächsten in der Reihe wechselt}
}
\newglossaryentry{Phase}{
	name=Phase,
	description={Eine Phase ist Teil eines Zuges. Es wird beim aktiven Spielen zwischen der
	Aktions- und Kaufphase unterschieden. Die Aufräumphase wird automatisch ohne Interaktion
	des Spielers ausgeführt}
}
\newglossaryentry{FAQ}{
	name=FAQ,
	description={Hier werden häufig gefragte Fragen angezeigt}
}
\newglossaryentry{CodeReview}{
	name=Code-Review,
	description={Damit ein Pull-Request durchgeführt werden kann, müssen mindestens zwei Code-Reviews
	durchgeführt werden. Dabei wird der geschriebene Code von zwei Personen unabhängig
	voneinander auf Fehler oder Verbesserungsmöglichkeiten kontrolliert}
}
\newglossaryentry{PullRequest}{
	name=Pull-Request,
	description={Ein Pull-Request wird eröffnet, wenn eine bestimmte Aufgabe \bzw User Story
	abgeschlossen wurde und der geschriebene Code mit der bereits bestehenden Code-Basis
	zusammengeführt werden soll}
}
\newglossaryentry{Framework}{
	name=Framework,
	description={Ein Framework ist eine Software, die noch keine fertige Implementierung
	bereitstellt, jedoch einen Rahmen anbietet, der bei der Umsetzung eine Hilfestellung bietet}
}
\newglossaryentry{KI}{
	name=Künstliche Intelligenz,
	description={Als Künstliche Intelligenz oder auch Bots werden Spieler bezeichnet, die nicht
	direkt durch einen angemeldeten Benutzer bedient werden, sondern computergesteuert nach
	verschiedenen, vordefinierten Mustern in einem Spiel agieren}
}
\newglossaryentry{Hamburger-Button}{
	name=Hamburger-Button,
	description={Ein umgangssprachlicher Name für ein Icon mit drei waagerechten, parallel zueinander platzierten Strichen. Es symbolisiert eine Menüliste}
}
\newglossaryentry{Unterseiten}{
	name=Unterseiten,
	description={Anderes Wort für die Tabs die geöffnet werden, wenn man z.B. die Optionen öffnet}
}
\newglossaryentry{Lombok}{
	name=Lombok,
	description={Lombok ist ein Java Plugin, welches das Setzen von Gettern und Settern im Code automatisiert}
}
\newglossaryentry{AFK}{
	name=AFK,
	description={AFK ist eine Abkürzung für Away From Keyboard und steht dafür, dass eine Person 
	zur Zeit nicht am PC ist. Für andere Spieler wird dies durch Untätigkeit wahrgenommen}
}
\newglossaryentry{CodeSmells}{
	name=Code-Smells,
	description={Code-Smells bezeichnen Fehler, die eine Überarbeitung des Codes nahelegen. Hierbei handelt es sich oft um funktionierenden Code mit schlechter Struktur, beispielsweise werden ungünstige Namen für Variablen, Klassen und Methoden gewählt oder Code wird per Copy-and-Paste unnötig oft dupliziert}
}

\makeglossaries
\glsaddallunused