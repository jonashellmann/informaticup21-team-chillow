\chapter{Weiterentwicklung}
\label{ch:weiterentwicklung}

Wichtig war bei dem Projekt auch, stets sicherzustellen, dass das Projekt möglichst leicht weiterentwickelt werden kann.
Was wir dazu unternommen haben, soll in diesem Kapitel erläutert werden, sowohl bei der Implementierung des Quell-Codes,
als auch bei der Entwicklung des die Umsetzung dieses Projektes umfassenden Workflows.

\section{Erweiterbarkeit des Codes}
\label{sec:erweiterbarkeit}

An vielen Stellen im Code haben wir durch die Nutzung allgemeiner Oberklassen eine leichte Austauschbarkeit
gewährleistet.
Die Verwendung von abstrakten Oberklassen durch Nutzung der \Code{abc}-Python-Bibliothek war notwendig, da Python
als Programmiersprache keine Interfaces anbietet. \\

Durch diese Entscheidung war es durch Dependency Injection möglich, zentral an einer Stelle in \Code{main.py}
festzulegen, welche spezifische Implementierung der allgemeinen Oberklasse verwendet werden soll.
Angewendet wurde dieses Vorgehen an oberster Stelle bei der \Code{Connection} und der zu verwendenden Oberfläche, aber
auch beim Konvertieren zwischen einer String-Repräsentation und dem Modell im \Code{DataLoader} und \Code{DataWriter}
ließe sich prinzipiell auch sehr leicht ein Umstieg von JSON auf ein beliebiges anderes Format ermöglichen.
Auch ein Austausch der zu verwendenden \ac{KI}-Klasse lässt sich sehr einfach konfigurieren.

\section{Einsatz von PRs im Git-Workflow}
\label{sec:git-workflow}

Wir haben uns dazu entschieden, als Versionsverwaltungs-Tool Git einzusetzen und auf Github als Plattform zu bauen.
Github ermöglicht die Konfiguration, das Pushen auf den Haupt-Branch, welcher in unserem Fall der \Code{main}-Branch war,
zu unterbinden.
So kann kein Code durch einen versehentlichen Commit den aktuellen Produktiv-Code in seiner Funktionalität stören.
Stattdessen können Änderungen in diesen Branch nur durch \ac{PR}s in den Haupt-Branch gemergt werden. \\

Es gab somit für jede logische Einheit für Code-Änderungen einen neuen Branch, auf dem diese entwickelt und getestet
wurden, bevor eine Überführung in \Code{main} möglich war.
Es wäre möglich gewesen, die Aufgaben als sogenannte Issues zu pflegen und jeden Branch mit einem Issue zu verknüfen,
allerdings haben wir dies nicht genutzt, da wir uns durch regelmäßige Absprachen auch so einen Überblick über die
als nächstes zu erledigenden Aufgaben ohne ein Ticket-System machen konnten. \\

Für \ac{PR}s wurden dann Kriterien festgelegt, die erfüllt sein müssen, um einen Merge durchführen zu können.
Es wird automatisch von Github kontrolliert, ob mögliche Konflikte beim Mergen auftreten können.
Falls dies so sein sollte, ist es notwendig, diese zuerst manuell zu beheben.
Weiterhin haben wir eingestellt, dass mindestens ein Review notwendig ist.
Da wir als Zweiergruppe an dem Projekt gearbeitet haben, konnten wir so sicherstellen, dass jeder zu jeder Zeit einen
Überblick über den aktuellen Stand hat und jeder Code einem Review unterzogen wurde.

\section{Nutzung von Github Actions}
\label{sec:github-actions}

Als letzter Aspekt, der einen Merge potentiell verhindern konnte, wurde eine sogenannte Github Action bei dem Öffnen
eines \ac{PR}s und bei dem Pushen auf einen Branch mit einem bereits geöffneten \ac{PR} ausgeführt. \\

Eine solche Aktion wird nicht manuell in den Einstellungen hinterlegt, sondern nach dem
Configuration-as-Code-Paradigma in einer Text-Datei verwaltet.
Dazu muss lediglich in einem Unterordner \texttt{./.github/workflows} ausgehend vom Hauptverzeichnis des Repositorys
eine Datei im YAML-Format abgelegt werden. \Vgl{github-actions-1} \Vgl{github-actions-2}. \\

Die für unser Projekt verwendete Konfiguration wird in \Anhang{lst:yaml-github-action} gezeigt.
Hier werden nach der Installation aller notwendigen Packages zuerst eine Kompilierung und Code-Analyse über den
Quellcode gefahren und auf Probleme hin untersucht und im Anschluss werden alle Tests ausgeführt.
Bei Kompilier-Fehler oder fehlschlagenden Tests schlägt auch diese Aktion fehl und verhindert einen Merge. \\

So wird durch Continious Integration automatisch eine Kontrolle vollzogen, die sicherstellt, dass durch Änderungen
kein bestehende Logik beschädigt wird.
Dieses gibt dem Entwickler eine zusätzliche Sicherheit und verringert die Risiken eines Merges.