\chapter{Lösungsansatz}
\label{ch:loesungsansatz}

% Implementierung mehrerer AIs
% Ggf. Vorgehen beschreiben, wie die beste KI ermittelt wurde
% Probleme bei Erstellung eines Baums, da bei Vorausberechnung viele Kombinationen entstehen
% Probleme für Reinforcement Learning, da Testdaten schwierig zu generieren sind

\section{Selbstlernende KI mithilfe von Trainingsdaten}
\label{sec:selbstlernende-ki-trainingsdaten}

Zunächst einmal haben wir uns die Frage gestellt, ob eine \acs{KI} mithilfe von Trainingsdaten lernen soll.
Dieser Ansatz bringt jedoch das Problem mit sich, gute Trainingsdatensätze besitzen zu müssen. 
Bei einer maximalen Anzahl von sechs Spielern im Spiel \texttt{spe\_ed} und 5 unterschiedlichen möglichen Aktionen gibt
es sehr viele Möglichkeiten wie ein Spiel verlaufen kann. 
Dabei entsteht das Problem zu beurteilen, welche Spielsituationen und welche Spielverläufe als Trainingsdaten gut geeignet sind, 
sodass die gesamte Komplexität des Spiels in unseren Trainingsdaten abgebildet wird.
Aufgrund dessen haben wir uns dafür entschieden, zunächst Lösungsansätze ohne lernende \acs{KI}s auszuprobieren, bevor wir eine \acs{KI} mittels Trainingsdaten lernen lassen. 
Diese Strategie hatte für uns den Vorteil mithilfe einfacherer Lösungsansätze die Komplexität und auftretenden Probleme im Spielverlauf besser kennen zu lernen.


\section{Lösungsansatz nicht lernende KI}
\label{sec:loesungsansatz-nicht-lernende-KI}

Bei dem Ansatz Strategien fest in Code zu implementieren hatten wir mehrere unterschiedliche Ideen, die nachfolgend
