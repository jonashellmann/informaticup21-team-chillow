\chapter{Lösungsansatz}
\label{ch:loesungsansatz}

% Implementierung mehrerer AIs
% Ggf. Vorgehen beschreiben, wie die beste KI ermittelt wurde
% Probleme bei Erstellung eines Baums, da bei Vorausberechnung viele Kombinationen entstehen
% Probleme für Reinforcement Learning, da Testdaten schwierig zu generieren sind


\section{Selbstlernende KI mithilfe von Trainingsdaten}
\label{sec:selbstlernende-ki-trainingsdaten}

Zunächst einmal haben wir uns die Frage gestellt, ob eine \acs{KI} mithilfe von Trainingsdaten lernen soll.
Dieser Ansatz bringt jedoch das Problem mit sich, gute Trainingsdatensätze besitzen zu müssen.
Bei einer maximalen Anzahl von sechs Spielern im Spiel \texttt{spe\_ed} und 5 unterschiedlichen möglichen Aktionen gibt
es sehr viele Möglichkeiten wie ein Spiel verlaufen kann.
Dabei entsteht das Problem zu beurteilen, welche Spielsituationen und welche Spielverläufe als Trainingsdaten gut
geeignet sind, sodass die gesamte Komplexität des Spiels in unseren Trainingsdaten abgebildet wird.
Aufgrund dessen haben wir uns dafür entschieden, zunächst Lösungsansätze ohne lernende \acs{KI}s auszuprobieren, bevor
wir eine \acs{KI} mittels Trainingsdaten lernen lassen.
Diese Strategie hatte für uns den Vorteil mithilfe einfacherer Lösungsansätze die Komplexität und auftretenden Probleme
im Spielverlauf besser kennen zu lernen.


\section{Lösungsansatz nicht-lernende-KI}
\label{sec:loesungsansatz-nicht-lernende-KI}

Bei dem Ansatz, Strategien fest in Code zu implementieren, hatten wir mehrere unterschiedliche Ideen,
die nachfolgend beschrieben werden.
Ziel war es hierbei, Teilprobleme zu erkennen und zu lösen, mit der Intuition diese unterschiedlichen \acs{KI}s
kombinieren zu können.


\section{RandomAI}
\label{sec:random-ai}

Die \Code{RandomAI} ist unsere erste lauffähige KI gewesen und unser Maß für die einfachste \acs{KI}.
Diese stellt zwar keinen wirklichen Lösungsansatz dar, diente jedoch als Einstieg und um erste Probleme zu erkennen.


\section{NotKillingItselfAI}
\label{sec:notkillingitself-ai}

Durch die \Code{RandomAI}, die zufällige Aktionen ausführt, ist uns das grundlegende Problem aufgefallen, dass sich die
\acs{KI} selber tötet.
Folglich wird die Minimal- und Maximalgeschwindigkeit überschritten, das Spielfeld verlassen oder in vorhandene Spuren
gefahren, mit denen die Kollision vermeidbar gewesen wäre.
Daher haben wir uns dafür entschieden, die \Code{NotKillingItselfAI} zu implementieren, die aus allen
Aktionen eine zufällige Aktion auswählt, die sie nicht direkt verlieren lässt.
Hierbei bleiben mögliche Aktionen der Gegenspieler zunächst unberücksichtigt.\\


\section{PathfindingAI}
\label{sec:pathfinding-ai}

Durch die Implementierung der \Code{NotKillingItselfAI} fiel auf, dass die \acs{KI} sich zwar nicht mehr im unmittelbar
folgenden Zug tötete, jedoch häufig in Sackgassen läuft.
Folglich haben wir einen Lösungsansatz gesucht, welcher das Betreten von Sackgassen verhindert.
Genutzt wurde die Basis der \Code{NotKillingItselfAI}, sodass nur zwischen überlebenden Aktionen gewählt wird.
Um die Aktionen zu finden, die nicht in eine Sackgasse läuft, haben wir uns für einen Lösungsansatz zum finden von
Pfaden entschieden.
Dazu wird eine konfigurierbare Anzahl Koordinaten generiert, auf denen sich bisher kein Spieler befindet.
Anschließend wird für jede Aktion geprüft, zu wie vielen der Koordinaten, nach der Ausführung der Aktion noch ein
möglicher Pfad existiert. \todo{In einem Bild verdeutlichen warum das ein Bewertungskriterium darstellt}
Die Aktion, die die höchste Anzahl Koordinaten erreichen kann, führt mit höchster Wahrscheinlichkeit
nicht in eine Sackgasse und wird ausgewählt.
Bei diesem Lösungsansatz arbeiten wir lediglich mit der Warscheinlichkeit, dass wir nicht in eine Sackgasse laufen.
Dies liegt dem Problem zugrunde, dass wir die Implementierung eines Algorithmus zum tatsächtichen Erkennen von
Sackgassen oder abgesperrten Gebieten als schwieriger erachteten.


\section{SearchTreeAI}
\label{sec:searchtree-ai}

Bei der \Code{PathfindingAI}, als auch der \Code{NotKillingItselfAI} gibt es weiterhin ein bestehendes
Problem.
Die \acs{KI} beachtet die möglichen Aktionen der Gegenspieler nicht.
Ziel der \Code{SearchTreeAI} ist es, dass die die \Code{NotKillingItselfAI}, falls möglich, auch nur Aktionen berechnet,
bei denen bereits alle möglichen Gegnerischen Aktionen und dessen Auswirkungen berücksichtigt werden.
Unser Lösungsansatz für dieses Problem ist es, mithilfe eines Suchbaums, alle möglichen gegnerischen
Aktions-Kombinationen zu prüfen und einen Teilbaum zu finden, bei dem vorausgesagt werden kann, dass
der Spieler nicht im nächsten Zug stirbt.
Der Vorteil eines Suchbaums ist es, dass wir diesem eine
beliebige Tiefe mitgeben können, um entsprechend der Tiefe viele zukünftige Züge vor zu berechnen.
\todo{Besser ausformulieren, Problem der Komplexität benennen und Beispielskizze}
