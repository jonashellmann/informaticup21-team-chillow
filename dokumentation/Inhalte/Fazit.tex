\chapter{Fazit}
\label{ch:fazit}

\section{Einschätzung unserer Lösung}
\label{sec:einschaetzung}

Insgesamt sind wir mit unserem Ergebnis nach Abschluss des Projektes sehr zufrieden.
Es wurde eine Software entwickelt, die leicht erweitert werden kann und deren Schichten voneinander sauber getrennt
sind.
Bei der Entwicklung wurde auf die Berücksichtigung bewährter Standards wie den Einsatz von Versionsverwaltung oder
Continuous Integration geachtet, es konnte vorher in der Universität erlerntes Wissen eingebracht werden und neue
Erkenntnisse gewonnen werden. \\

Im Hinblick auf die Stärke der abgegebenen \ac{KI} sind wir ebenfalls davon überzeugt, eine solide Leistung erzielen zu
können.
Mit der Kombinationen verschiedener Ansätze ist eine Vorausberechnung gegnerischer Züge und entsprechendes Reagieren
darauf genauso möglich wie das Verhindern von Sackgassen durch das Finden von Pfaden auf dem Spielfeld zu beliebig
ausgewählten Punkten.
Allerdings wurde in einem internen Turnier der Mannschaften der Universität Oldenburg auch deutlich,
dass andere Taktiken möglicherweise trotzdem stärker sein könnten und unsere im Vergleich lediglich mittelmäßig
abgeschnitten hat.
Unsere \ac{KI} wurde entwickelt, möglichst sichere Entscheidungen für die nächsten x Spielzüge zu treffen, was durch aggressive \ac{KI}s im Turnier ausgenutzt werden konnte.
In dem Turnier wurden jedoch nur drei Spiele ausgeführt, sodass die dort erzielte Gewinnquote nicht repräsentativ ist. 
Des Weiteren waren wir zu der Zeit noch
nicht am Ende der Entwicklung, weswegen wir trotzdem optimistisch sind und auf das Ergebnis vom InformatiCup
sehr gespannt sind.

\subsection{Umsetzung von optionalen Erweiterungen}
\label{subsec:optionale-erweiterungen}

Der geforderte Umfang der Aufgabenstellung \Vgl{informaticup21-aufgabe} wurde in unserem Lösungsvorschlag an einigen
Stellen durch zusätzliche Erweiterungen ergänzt.
Diese wurden in der Dokumentation in vorangegangenen Abschnitten bereits erläutert, sollen an dieser Stelle aber noch
einmal zusammenfassend erwähnt werden. \\

Es wurden zwei verschiedene \ac{UI}s für eine konsolenbasierte Ausgabe und eine grafische Darstellung des Spiels in
einer Desktop-Anwendung implementiert, auf deren Entwicklung in \Kapitel{subsec:oberflaeche-konsole} und
\Kapitel{subsec:oberflaeche-pygame} eingegangen wurde.
Diese Oberflächen erlauben neben der automatischen Berechnung der nächsten Aktion durch eine \ac{KI} mithilfe der im
\Kapitel{ch:benutzerhandbuch} genannten Konfigurationen auch die Interaktion eines menschlichen Spielers mittels
Benutzereingaben. \\

Zusätzlich zu der geforderten Aufgabe, "`ein Programm zu schreiben, das selbstständig (also ohne menschliche
Unterstützung) das Spiel spe\_ed spielen kann"' \Vgl{informaticup21-aufgabe}, welches lediglich als Client fungiert und
mit einem Server kommuniziert, wurde die gesamte Logik des Servers wie in \Kapitel{sec:offline-implementierung}
beschrieben nachgebildet.
Dadurch wird ermöglicht, in den bereitgestellten \ac{UI}s auch offline Spiele der \ac{KI}s durchzuführen. \\

Auf diese Möglichkeit wurde aufgebaut, indem eine Erweiterung bereitgestellt wird, die es ermöglicht, eine beliebige
Anzahl an Spielen mit zufällig generierten Spieleinstellungen zu erzeugen.
Hierauf wurde in \Kapitel{sec:auswahl-strategie} eingegangen.
Um auf die Ergebnisse dieser Simulationen zugreifen zu können, wurde ein Entity-Relationship-Modell entworfen, das in
ein relationales Datenbank-Modell überführt wurde.
Auf Grundlage dieses Modells wird bei der Simulation eine SQLite3-Datenbank verwendet und befüllt, aus der mithilfe von
SQL-Statements Informationen zu der Stärke einer \ac{KI} gewonnen werden können. \\

\section{Reflexion des Wettbewerbs}
\label{sec:reflexion}

Der Wettbewerb war aus unserer Sicht eine sehr gute Möglichkeit, sich im Rahmen des Studiums mit interessanten Themen
auseinanderzusetzen und diese teilweise auch praktisch umzusetzen.
Dazu zählen \ua Python als Programmiersprache, eine Einarbeitung in maschinelles Lernen und die Konzeption einer
guten, prädiktiven Strategie für ein Spiel mit einem grundsätzlich einfach zu verstehendem Regelwerk, das allerdings
bei der Vorhersage von Spielzügen durch die exponentiell schnell steigende Anzahl von Möglichkeiten sehr komplex wird.
\\

Auch der kompetitive Gedanke dieser Ausgabe des InformatiCups, dass einen Lösungsvorschlag entwickelt werden sollte,
der mit den anderen eingereichten Projekten mithalten \bzw gegen diese gewinnen kann, war sehr interessant.
