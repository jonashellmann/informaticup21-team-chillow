\chapter{Fazit}
\label{ch:fazit}

\section{Einschätzung unserer Lösung}
\label{sec:einschaetzuung}

\todo{Kapitel ausformulieren}

\subsection{Umsetzung von optionalen Erweiterungen}
\label{subsec:optionale-erweiterungen}

% Verschiedene Oberflächen, Offline-Spiel, Evaluation inkl. DB
\todo{Kapitel ausformulieren}

\section{Reflexion des Wettbewerbs}
\label{sec:reflexion}

Der Wettbewerb war aus unserer Sicht eine sehr gute Möglichkeit, sich im Rahmen des Studiums mit interessanten Themen
auseinanderzusetzen und diese teilweise auch praktisch umzusetzen.
Dazu zählen \ua Python als Programmiersprache, eine Einarbeitung in maschinelles Lernen und die Konzeption einer
guten, prädiktiven Strategie für ein Spiel mit einem grundsätzlich einfach zu verstehendem Regelwerk, das allerdings
bei der Vorhersage von Spielzügen durch die exponentiell schnell steigende Anzahl von Möglichkeiten sehr komplex wird.
\\

Auch der kompetitive Gedanke dieser Ausgabe des InformatiCups war sehr interessant, sodass man immer auch im
Hinterkopf hatte, einen Lösungsvorschlag zu entwickeln, der mit den anderen eingereichten Projekte mithalten \bzw
gegen diese gewinnen kann.
