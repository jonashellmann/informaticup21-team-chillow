\chapter{Benutzerhandbuch}
\label{ch:benutzerhandbuch}

Das Benutzerhandbuch soll eine Anleitung darstellen, wie die eingereichte Lösung installiert und ausgeführt werden kann.
Dafür wird zwischen der Verwendung von Docker oder einer manuellen Installation unterschieden.

\section{Installation}
\label{sec:installation}

Zur Verwendung dieses Projektes muss es lokal heruntergeladen werden, entweder durch Klonen des Repositorys oder durch
einen Download als ZIP-Datei.
Das Projekt kann unter folgendem Link eingesehen werden:
\url{https://github.com/jonashellmann/informaticup21-team-chillow}

\subsection{Docker}
\label{subsec:docker}

Falls Sie Docker auf Ihrem Rechner installiert haben, lässt sich für das Projekt aufgrund des vorhandenen
\Code{Dockerfile}s mit folgendem Befehl ein neuer Container erstellen:

\begin{verbatim}
docker build -t informaticup21-team-chillow .
\end{verbatim}

Dieser Container kann mit folgendem Befehl gestartet werden, wobei die URL zum spe\_ed-Server und der API-Key
entsprechend angepasst werden müssen:

\begin{verbatim}
docker run -e URL=SERVER_URL -e KEY=API_KEY informaticup21-team-chillow
\end{verbatim}

In der Konsole des Docker-Containers lässt sich dann der Spiel-Verlauf nachvollziehen.

\subsection{Manuelle Installation}
\label{subsec:manuelle-installation}

Neben der Docker-Installation kann das Projekt auch eigenständig gebaut werden.
Dafür ist erforderlich, dass neben Python in der Version 3.8 auch Poetry als Build-Tool installiert ist.

Die erforderlichen Abhängigkeiten lassen sich anschließend mittels \Code{poetry install} installieren.

Um ein Spiel mit einer simplen grafischen Oberfläche zu starten, in dem gegen die implementierte \ac{KI} gespielt
werden kann, genügt der Befehl \Code{python ./main.py}.
Wenn gegen eine andere \ac{KI} gespielt werden soll als die, für die wir uns am Ende entschieden haben, muss dies im
\Code{OfflineController} bei der Erstellung des initialen Spiels manuell angepasst werden.

Um ein Online-Spiel der KI auf dem Server zu starten, müssen folgende Umgebungsvariablen verwendet werden, die im
Docker-Container automatisch gesetzt \bzw als Parameter übergeben werden:

\begin{itemize}
    \item \Code{PLAY\_ONLINE=TRUE}
	\item \Code{URL=[SERVER\_URL]}
	\item \Code{KEY=[API\_KEY]}
\end{itemize}

Mittels der Umgebungsvariable \Code{DEACTIVATE\_PYGAME} kann entschieden werden, ob eine grafische Oberfläche benutzt
werden soll oder die Ausgabe wie im Docker-Container über die Konsole erfolgt.

\section{Benutzung}
\label{sec:benutzung}

Wenn das Programm im Online-Modus gestartet wird, ist keine weitere Eingabe des Benutzers zu tätigen.
Sobald der Server das Spiel startet, kann entweder auf der Konsole oder in der grafischen Oberfläche der Spielverlauf
nachvollzogen werden.
Hier muss die Umgebungsvariable \Code{PLAY\_ONLINE} auf \Code{TRUE} gesetzt werden.

Bei einer Ausführung im Offline-Modus wird - je nach manueller Anpassung im \Code{OfflineController} - auf eine
Eingabe von einem oder mehreren Spielern gewartet, bis die nächste Runde des Spiels gestartet wird.
Der \Tabelle{tab:eingaben-oeberflaeche} kann entnommen werden, mit welchen Eingaben eine Aktion ausgeführt werden kann.
Die Umgebungsvariable \Code{PLAY\_ONLINE} muss für diesen Modus auf \Code{FALSE} gesetzt werden.

\begin{table}[htb]
    \centering
    \begin{tabular}{|l|c|c|c|c|c|}
        \hline
         & \textbf{turn\_right} & {\textbf{turn\_left}} & \textbf{speed\_up} & \textbf{slow\_down} & \textbf{change\_nothing} \\ \hline
        \textbf{Konsole} & r & l & u & d & n \\ \hline
        \textbf{Grafische Oberfläche} & → & ← & ↑ & ↓ & Leertaste \\ \hline
    \end{tabular}
    \caption{Steuerung der Oberflächen}
    \label{tab:eingaben-oeberflaeche}
\end{table}

Darüber hinaus ist eine Offline-Simulation mehrerer Spiele hintereinander möglich, in dem \ac{KI}s mit zufälliger
Konfiguration auf einem Spielfeld mit zufälliger Größe gegeneinander antreten, um die bestmögliche \ac{KI} zu ermitteln.
Dazu ist es notwendig, dass zusätzlich zum normalen Offline-Spiel die Umgebungsvariable \Code{AI\_EVALUATION\_RUNS} auf
eine Zahl größer als Null gesetzt wird.
Mit der Umgebungsvariable \Code{AI\_EVALUATION\_DB\_PATH} kann statt dem Standardwert auch individuell der Pfad zu einer
SQLite3-Datenbank festgelegt werden.