\chapter{Implementierung}
\label{ch:implementierung}

Den Einstieg in das Programm stellt die Datei \Code{main.py} dar.
Hier wird entschieden, ob ein Online- oder Offline-Spiel wie schon in den vorherigen Kapiteln beschrieben gestartet
werden soll.
Die Implementierungen dieser beiden Spielvarianten sollen in diesem Kapitel beschrieben werden, wobei der Fokus auf
die Online-Verbindung gerichtet ist, da es sich hierbei um die Umsetzung der eigentlichen Aufgabenstellung handelt.

\section{Modellierung des Spiels}
\label{sec:modellierung}

% UML-Klassendiagramm des Models

\section{Implementierung des Online-Spiels}
\label{sec:online-implementierung}

Um eine Verbindung zu dem \Code{spe\_ed}-Server aufbauen zu können, müssen die URL und ein gültiger API-Key vor dem
Start der Anwendung als Umgebungsvariablen gesetzt worden sein.
Diese Websocket-URL wird entsprechend modifiziert auch als Endpunkt zur Abfrage der Server-Zeit verwendet, auf deren
Nutzung nachfolgend noch eingegangen wird.

Die zum Start des Spiels öffentlich bereitgestellte Methode \Code{play()} ist in \Listing{lst:online-play} dargestellt.
Diese ist sehr simpel und ruft lediglich die private Methode \Code{\_\_play()} auf.
Hierbei handelt es sich um eine asychrone Methode, wie schon in der Methoden-Signatur deutlich wird.
Es wird mittels der Bibliothek \Code{asyncio} sichergestellt, dass diese asynchrone Methode vollständig verarbeitet
wurde, bevor der Kontrollfluss im Programm weiterläuft.
Sobald dies der Fall ist, ist das Spielende eingetreten und die Oberfläche wird deinitialisiert.

\lstinputlisting[label=lst:online-play,language=Python,caption=\Code{play()}-Methode der \Code{OnlineConnection}]
{./Dokumente/OnlineConnection-play.txt}

In der im vorherigen Absatz bereits erwähnten Methode \Code{\_\_play()} in der \Code{OnlineConnection} wird eine
Websocket-Verbindung zum Server aufgebaut.
Anschließend wird in einer Endlosschleife die Logik zur Ausführung eines einzelnen Spielzugs ausgeführt.
Wie ein solcher Spielzug abläuft, ist in \Anhang{fig:sequenzdiagramm-spielzug} in Form eines Sequenzdiagrammes
nachvollziehbar und die Umsetzung in Python-Code kann zusammen mit dem Verbindungaufbau \Listing{lst:online-__play}
entnommen werden.

\lstinputlisting[label=lst:online-__play,language=Python,caption=\Code{\_\_play()}-Methode der \Code{OnlineConnection}]
{./Dokumente/OnlineConnection-__play.txt}

\subsection{Einlesen des Spiel-Zustands}
\label{subsec:einlesen-spielzustand}

\subsection{Ermitteln der besten Aktion}
\label{subsec:ermitteln-aktion}

\subsection{Übergeben der Aktion an Web-API}
\label{subsec:uebergabe-aktion}

\section{Implementierung des Offline-Spiels}
\label{sec:offline-implementierung}

\section{Bereitstellung einer Oberfläche}
\label{sec:bereitstellung-oberflaeche}

\subsection{Darstellung des Spiels in der Konsole}
\label{subsec:oberflaeche-konsole}

\subsection{Nutzung von PyGame als grafische Oberfläche}
\label{subsec:oberflaeche-pygame}