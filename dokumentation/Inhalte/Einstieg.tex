\chapter{Einstieg in das Projekt}
\label{ch:planung}

Bevor mit der Implementierung der \ac{KI} begonnen werden konnte, mussten zuerst noch einge andere Punkte geklärt
\bzw erledigt werden.
Diese ersten Schritte werden im folgenden erläutert.

\section{Auswahl der Programmiersprache}
\label{sec:auswahl-programmiersprache}

Zu beginn war zu klären, mit welcher Programmiersprache dieser Lösungsvorschlag umgesetzt werden soll.
Die Wahl ist dabei schnell auf Python gefallen, obwohl beide Gruppenmitglieder hiermit noch keinerlei Erfahrung
aufweisen konnte.
Der Grund für diese Entscheidung liegt neben dem starken Interessen an dem Kennenlernen einer neuen Programmiersprache
auch an der bereits sehr hohen und immer noch steigenden Popularität der Programmiersprache und der damit verbundenen
zukünftigen Wichtigkeit. \Vgl{tiobe} \Vgl{pypl}
Hinzu kommt, dass wir vor Beginn der Implementierung anhand der Aufgabenstellung das Potenzial für den Einsatz von
Machine Learning gesehen haben und Python in diesem Bereich oft empfohlen wird. \Vgl{springboard} \Vgl{towardsscience}

\subsection{Einsatz von Poetry als Build-Tool}
\label{subsec:poetry}

% Vorteile eines Build-Tools und Poetry im Speziellen
% Verwaltung von Abhängigkeiten mit Poetry

\section{Erstellung eines lauffähigen Projekts}
\label{sec:erstellung-projekt}

\subsection{Entwicklung eines Dockerfile}
\label{subsec:dockerfile}

Nachdem die Nutzung von Python feststand, haben wir zum Start ein minimales Python-Skript erstellt, das lediglich die
an den Docker-Container übergebenen Parameter für die Server-URL und den API-Key ausgibt.
Zwar mussten bis hierhin noch keine Abhängigkeiten hinzugefügt werden, aber bei der Konzeption des Dockerfiles sollte
bereits die Installtion zusätzlicher Bibliotheken berücksichtigt werden.
Mit der Nutzung des Standard-Python-Containers von Docker Hub wird bereits ein vorgefertigter Container bereitgestellt,
in dem Python und Pip installiert ist.
Anschließend wird mittels Pip die Installation von Poetry durchgeführt.
Dieses Tool wiederum bietet die Möglichkeit, die verwalteten Abhängigkeiten in Form einer
\texttt{requirements.txt}-Datei zu exportieren, welche dann von Pip eingelesen werden kann, um die Bibliotheken zu
installieren.
Später sind noch Umgebungsvariablen zur Steuerung des Programm-Ablaufs hinzugekommen, sodass das Dockerfile
letztendlich wie in \Listing{lst:dockerfile} dargestellt aussieht.

\lstinputlisting[label=lst:dockerfile,language=Java,caption=Dockerfile zum Erstellen eines lauffähigen Containers]
{../Dockerfile}

\section{Nachstellung des Spiels}
\label{sec:nachstellung-spiel}

% Spiel nachprogrammiert, um selbst offline gegen die KIs spielen zu können
% Vorgriff auf spätere Kapitel: Zentrale Funktionalitäten konnten später wiederverwendet werden