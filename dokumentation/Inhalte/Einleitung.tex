\chapter{Einleitung}
\label{ch:einleitung}

In dieser Dokumentation wird der Lösungsentwurf vom Team Chillow der Universität Oldenburg für den InformatiCup 2021,
der von der Gesellschaft für Informatik organisiert wird, beschrieben.
Das Team besteht aus den Mitgliedern Florian Trei und Jonas Hellmann, die zum Zeitpunkt des Wettbewerbs im fünften
Semester in den Studiengängen B. Sc. Informatik \bzw B. Sc. Wirtschaftsinformatik eingeschrieben sind.
Das Repository mit dem Quellcode zu der hier beschriebenen Lösung ist unter folgendem Link abrufbar:
\url{https://github.com/jonashellmann/informaticup21-team-chillow} \\

Die Aufgabenstellung\footnote{Eine genauere Beschreibung lässt sich in folgendem PDF-Dokument finden:
\url{https://github.com/informatiCup/InformatiCup2021/blob/master/spe\_ed.pdf}} sieht eine Implementierung des Spiels
\texttt{spe\_ed} vor.
Hierbei steuern bis zu sechs Spieler rundenbasiert eine Figur, die besuchte Felder markiert und bei einer Kollision
mit einem bereits markierten Feld oder beim Verlassen des Spielfelds verliert.
Ziel ist es, eine eigenständig spielende \ac{KI} zu programmieren, die möglichst viele Spiele gewinnt.